%\vspace*{\fill}
\vspace*{1.0in}

\begin{center}
\begin{large}
{\bf Abstract}
\end{large}
\end{center}

Shape Analysis refers to a class of techniques used to analyze heap data structures. Several algorithms have been proposed in literature
about shape analysis. These algorithms differ in the trade off they have between speed and accuracy.
    
  In this thesis we have implemented and proposed enhancements for a field sensitive shape analysis approach. These enhancements involve modification of data flow values, and intelligent
way of storing the same which makes the analysis more precise and memory efficient. This work also proposes an analysis namely \emph{Subset Based Analysis}
which infers more precise shapes depending upon which field pointers are actually accessed in a function. To handle functions we have developed
a new method for interprocedural analysis called \emph{Shape Sensitive Analysis}. This is a middle way between Context Sensitive and Context Insensitive Interprocedural Analysis.
We have implemented this analysis as a plugin for gcc 4.5.0 and the results for the same are presented. 
