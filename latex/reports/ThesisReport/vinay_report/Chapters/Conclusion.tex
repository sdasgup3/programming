% \section{Conclusion}

In this report we have suggested several enhancements to Sandeep's work on Field Sensitive analysis
which ensure the correctness and increase the accuracy of the analysis.
Now after these changes the analysis is better than Ghiya's work in all those unit test cases described. 
We have also introduced a new analysis called subset based analysis which infers shape based on the subset of fields actually accessed inside
a function. This helps us inferring information like a function is traversing/accessing a tree substructure of a cyclic data structure. 
We also proposed a shape sensitive inter procedural
analysis which is in mid way of context sensitive and context insensitive analysis and could possibly balance the memory consumption and 
preciseness at the same time. We have also performed various 
optimizations with the aim of decreasing the memory consumption and time for completion.
The testing strategy used is exhaustive and has helped a lot in identifying the cases of safe and incorrect results.

There are some benchmarks where the results are not as good as Ghiya's, these are due to the
summarization of heap nodes due to the dummy statement. In the future we plan to work on
this issue.
The analysis has concerns over the amount of memory it takes even after the optimizations performed. We want
to address this concern by representing boolean equations in much efficient way. 
% The callstring approach is using
% more memory for this analysis, so we plan to find out some other context sensitive analysis which is
% memory efficient.

% \section{Future Work}
% Representation of boolean equations in a memory efficient
% Better memory efficient Context sensitive analysis
% Extend this for arrays of pointers


