\documentclass[10pt,twoside]{article}
\usepackage{url}

\newcommand{\doctitle}{%
Approximations for the Feedback Vertex Set Problem}

\pagestyle{myheadings}
\markboth{\hfill\doctitle}{\doctitle\hfill}

\bibliographystyle{siam}

\addtolength{\textwidth}{1.00in}
\addtolength{\textheight}{1.00in}
\addtolength{\evensidemargin}{-1.00in}
\addtolength{\oddsidemargin}{-0.00in}
\addtolength{\topmargin}{-.50in}

\hyphenation{in-de-pen-dent}

%\title{\textbf{\doctitle}\\
\title{\textbf{ CS533 Initial Project Proposal}}

\author{Dhawal Seth\thanks{Electronic address: \texttt{dseth3@illinois.edu}}
\qquad Karthik GR\thanks{Electronic address: \texttt{gooli2@illinois.edu}}
\qquad  Sandeep Dasgupta \thanks{Electronic address:
\texttt{sdasgup3@illinois.edu}}} 

\begin{document}

\thispagestyle{empty}

\maketitle

We want to use Charm++ \url{[http://charm.cs.uiuc.edu/]} adaptive runtime system to maximize performance of
parallel applications under a given power budget. Newer systems such as Intel
Sandybridge allows user to constrain the power consumption by its compute cores
and DRAM. This facilitates software controlled, optimized power allocation to
the compute nodes based on the application running on them. We want to pursue a
project in this direction.  

Following is a specific idea that we are planning to
explore: 
It has been empirically observed that under the same power cap,
different nodes yield different application performance. This can be due to
several design factors: difference in chip designs, different in component
assembly by the machine vendor, location of the node in the data center,
difference in component design such as fans, etc. This difference in the design
causes load imbalance across nodes despite same allocated power and equal
compute load. Charm++ is based on over-decomposition and allows dynamic object
migration across processes. We want to use this feature of Charm++ to achieve
load balance in the presence of such heterogeneity in the nodes.

%\nocite{*}
%\bibliography{CS533_project_proposal}

\end{document}
