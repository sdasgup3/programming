A large part of the definitions and terms used in this chapter are borrowed from aforementioned paper.At each program
point they compute three abstractions that work together to find shape information.For each heap pointer they approximate the 
attribute shape and direction,interference relationships are approximated for each pair of heap directed pointers.
Those three abstractions are given below

\begin{definition}
Given any heap-directed pointer $p$, the shape attribute p.shape is Tree, 
if in the data structure accessible from p there is a unique (possibly empty) access path 
between any two nodes (heap objects) belonging to it. It is considered to be DAG (directed acyclic graph), 
if there can be more than one path between any two nodes in this data structure, 
but there is no path from a node to itself (i. e, it is acyclic). 
If the data structure contains a node having a path to itself, p.shape is considered to be Cycle.
Note that as lists are special case of tree data structures, their shape is also considered as Tree.
\end{definition}

\begin{definition}
Given two heap directed pointers $p$ and $q$, the direction matrix $D$ captures the following 
relationships between them:
\begin{itemize}
\item $D[p,q] = 1 : $ An access path possibly exists in the heap, from the heap object pointed to by $p$,
to the heap object pointed to by $q$. In this case we simply say that the pointer $p$ has a path to 
pointer $q$.
\item $D[p,q] = 0 : $ No access path exists from the heap object pointed to by $p$ to the heap object pointed to by $q$. 
\end{itemize}
\end{definition}

\begin{definition}
Given two heap directed pointers $p$ and $q$, the direction matrix $I$ captures the following 
relationships between them:
\begin{itemize}
\item $I[p,q] = 1 : $ A common heap object can be possibly accessed starting from pointers $p$ and $q$.
In this case we state that pointers $p$ and $q$ can interfere.
\item $I[p,q] = 0 : $ No common heap object can be accessed starting from pointers $p$ and $q$.
In this case we state that pointers $p$ and $q$ do not interfere.
\end{itemize}
\end{definition}

Direction relationships are used to actually estimate the shape attributes, where the interference 
relationships are used for safely calculating direction relationships. 



\subsection*{Illustrative Example}

The direction and interference matrices are illustrated in Fig.~\ref{fig:relwork_1}.
Part (a) represents a heap structures at a program point, while parts (b) and (c) show the direction 
and interference matrices for it.  
\begin{figure}
\centering
\begin{tabular}{c@{$\qquad\qquad$}c}
\multicolumn{2}{c}{
\scalebox{0.80} { \includegraphics[scale=1]{Figure/figure_2}}
} \\
\multicolumn{2}{c}{
\scalebox{0.80}{ (a) Heap Structure }
} \\ \\
\scalebox{0.80} { \begin{tabular}{|c||c|c|c|c|c|c|}
\hline
$D$ & $p$ & $q$ & $r$ & $s$ & $t$ & $u$ \\ \hline \hline
$p$ & $1$ & $1$ & $0$ & $0$ & $0$ & $0$ \\ \hline 
$q$ & $0$ & $1$ & $0$ & $0$ & $0$ & $0$ \\ \hline 
$r$ & $0$ & $0$ & $1$ & $0$ & $0$ & $0$ \\ \hline 
$s$ & $0$ & $0$ & $1$ & $1$ & $1$ & $0$ \\ \hline 
$t$ & $0$ & $0$ & $0$ & $0$ & $1$ & $0$ \\ \hline 
$u$ & $0$ & $0$ & $1$ & $0$ & $0$ & $1$ \\ \hline 
\end{tabular}} 
& 
\scalebox{0.80} {\begin{tabular}{|c||c|c|c|c|c|c|}
\hline
$I$ & $p$ & $q$ & $r$ & $s$ & $t$ & $u$ \\ \hline \hline
$p$ & $1$ & $1$ & $0$ & $0$ & $0$ & $0$ \\ \hline 
$q$ & $1$ & $1$ & $0$ & $0$ & $0$ & $0$ \\ \hline 
$r$ & $0$ & $0$ & $1$ & $1$ & $0$ & $1$ \\ \hline 
$s$ & $0$ & $0$ & $1$ & $1$ & $1$ & $1$ \\ \hline 
$t$ & $0$ & $0$ & $0$ & $1$ & $1$ & $0$ \\ \hline 
$u$ & $0$ & $0$ & $1$ & $1$ & $0$ & $1$ \\ \hline 
\end{tabular}}  \\
\scalebox{0.80}{ (b) Direction Matrix}  & \scalebox{0.80} {(c) Interference Matrix}
\end{tabular}
\caption{Example Direction and Interference Matrices}
\label{fig:relwork_1}
\end{figure}


We now demonstrate how direction relationships help estimate the shape of the data structures.
In Fig.~\ref{fig:relwork_2}, initially we have both \p.\shape\ and \q.\shape\ as Tree. Further $D[q,p] == 1$, as there 
exists a path from \q\ to \p\ through {\tt next} link. The statement {\tt p$\rightarrow$prev = q}, sets up a path from
\p\ to \q\  through the {\tt prev} link. From direction matrix information we already know that a path exists
from \q\ to \p, and now a path is being set from \p\ to \q. Thus after the statement, $D[p,q] = 1$, $D[q,p] = 1$, \p.\shape\ = Cycle
and \q.\shape\ = Cycle.  

\begin{figure}
\centering
\scalebox{.80} {
\includegraphics[scale=1]{diagrams/Appendix_3.eps}
}
\caption{Example Demonstrating Shape Estimation}
\label{fig:relwork_2}
\end{figure}

\subsection*{Analysis of Basic Statements}
They have considered eight basic statements that can access or
modify heap data structures as listed in Fig.~\ref{fig:relwork_3}(a). Variables \p\ and \q and the field $f$ are
of pointer type, variable $k$ is of integer type, and $op$ denotes the $+$ and $-$ operations. The 
overall structure of the analysis is shown in Fig.~\ref{fig:relwork_3}(b). Given the direction and the interference 
matrices $D$ and $I$ at a program point x, before the given statement, they compute the matrices $D_n$ and $I_n$
at a program point y. Additionally, we have the attribute matrix A, where for a pointer $p$, $A[p]$ gives its shape attribute.
The attribute matrix after the statement is presented as $A_n$.

For each statement they compute the set of direction and interference relationships it kills and generates. Using these sets, the
new matrices $D_n$ and $I_n$ are computed as shown in Fig.~\ref{fig:relwork_3}(c). Note that the elements in the gen and kill sets are denoted as $D[p,q]$
for direction relationships, and $I[p,q]$ for interference relationships. Thus a gen set of the form $\{D[x,y], D[y,z]\}$, indicates that
the corresponding entries in the output direction matrix $D_n[x,y]$ and $D_n[y,z]$ should be set to one. We also compute the set of 
pointers $H_s$, whose shape  attribute can be modified by the given statement. Another attribute matrix $A_c$ is used to store the 
changed attribute of pointers belonging to the set $H_s$. The attribute matrix $A_n$ is then computed using the 
matrices $A$ and $A_c$ as shown in Fig.~\ref{fig:relwork_3}(c).  

Let $H$ be the set of pointers whose relationships/attributes are abstracted by the matrices $D$. $I$ and $A$. Further
assume that updating an interference matrix entry $I[\q,\p]$, implies identically updating the entry $I[\p,\q]$.   
\input{Chapters/relatedwork_fig_2.tex}

The actual analysis rules can be divided into three groups: (1) allocations, (2) pointer assignments, and 
(3) structure updates. Figure~\ref{fig:relwork_4} shows the gen and kill sets corresponding to each statement. 
\begin{figure}
\centering
\begin{tabular}{|l|c|}
\hline
1. {\tt p  = malloc();} &  
					$\begin{array}{lll}
						D\_kill\_set &=& \{D[p,s] \vert s \in H \wedge D[p,s]\}\ \cup\\
                                     &&   \{D[s,p] \vert s \in H \wedge D[s,p]\} \\
						I\_kill\_set &=& \{I[p,s] \vert s \in H \wedge I[p,s]\} \\
						D\_gen\_set &=& \{D[p,p]\} \quad I\_gen\_set = \{I[p,p]\} \\
						H_s &=& \{p\} \quad A_c[p] = Tree 	\\
					\end{array}$
								\\
\hline
\begin{tabular}{l}
2. {\tt p = q;} \\
3. {\tt p = \&(q$\rightarrow$f);} \\
4. {\tt p = q op k;} \\
\end{tabular}
 & 
				\begin{tabular}{l}
					Kill set same as that of {\tt p  = malloc();} \\
					$\begin{array}{lll}
						D\_gen\_set\_from &=& \{D[s,p] \vert s \in H \wedge s \not= p \wedge D[s,q]\} \\
						D\_gen\_set\_to &=& \{D[p,s] \vert s \in H \wedge s \not= p \wedge D[q,s]\} \\
						I\_gen\_set	&=& \{I[p,s] \vert s \in H \wedge s \not= p \wedge I[q,s]\}\ \cup \\
											&&  \{I[p,p] \vert I[q,q]\} \\
						D\_gen\_set	&=& D\_gen\_set\_from\ \cup D\_gen\_set\_to \\
						H_s &=& \{p\} \quad A_c[p] = A[q] 	\\
					\end{array}$
				\end{tabular}
								\\
\hline
5. {\tt p = NULL;} & 
				\begin{tabular}{l}
					Kill set same as that of {\tt p  = malloc();} \\
					$\begin{array}{lll}
						D\_gen\_set &=& \{\} \quad I\_gen\_set = \{\} \\
						H_s &=& \{p\} \quad A_c[p] = Tree 	\\
					\end{array}$
				\end{tabular}
								\\
\hline
6. {\tt p = q$\rightarrow$f;} & 
				\begin{tabular}{l}
					Kill set same as that of {\tt p  = malloc();} \\
					$\begin{array}{lll}
						D\_gen\_set\_from 	&=& \{D[s,p] \vert s \in H \wedge s \not= p \wedge I[s,q]\} \\
						D\_gen\_set\_to 	&=& \{D[p,s] \vert s \in H \wedge s \not= p \wedge s \not= q\ \wedge \\
                                            &&    D[q,s]\}\ \cup \{D[p,q] \vert A[q] = Cycle\}\ \cup \\
											&& \{D[p,p] \vert D[q,q]\} \\
						D\_gen\_set 		&=& D\_gen\_set\_from\ \cup D\_gen\_set\_to \\
						I\_gen\_set 		&=& \{I[p,s] \vert s \in H \wedge s \not= p \wedge I[q,s]\}\ \cup \\
											&&  \{I[p,p] \vert I[q,q]\} \\
						A_c[p] &=& A[q] 	\\
					\end{array}$
				\end{tabular}
								\\
\hline
7. {\tt p$\rightarrow$f = NULL;} & 
					$\begin{array}{lll}
						D\_kill\_set &=& \{\} \quad I\_kill\_set = \{\}\\
						D\_gen\_set 		&=& \{\} \quad I\_gen\_set  = \{\}\\
						A_c[p] &=& A[p]\ \forall p \in H  	\\
					\end{array}$
								\\
\hline
7. {\tt p$\rightarrow$f = q;} & 
				\begin{tabular}{l}
					Kill set same as that of {\tt p$\rightarrow$f = NULL;} \\
					$\begin{array}{lll}
						D\_gen\_set		&=& \{D[r,s] \vert r,s \in H \wedge D[r,p] \wedge D[q,s]\}  \\
						I\_gen\_set  	&=& \{I[r,s] \vert r,s \in H \wedge D[r,p] \wedge I[q,s]\} \\
					\end{array}$ \\ \\
					\underline{Pointer q already has a path to p, D[q,p] = 1} \\
					$\begin{array}{lll}
						H_s 			&=& \{s \vert s \in H \wedge (D[s,p] \vee D[s,q])\} \\
						D[q,p] &\Rightarrow& A_c[s] = Cycle\ \forall s \in H_s  	\\
					\end{array}$ \\ \\
					\underline{A[q] = Tree} \\
					$\begin{array}{lll}
						H_s 			&=& \{s \vert s \in H \wedge (D[s,p] \vee I[s,q])\} \\
						(\neg D[q,p] \wedge (A[q] = Tree)) &\Rightarrow& A_c[s] = A[s] \Join Dag\ \forall s \in H_s  	\\
					\end{array}$ \\ \\
					\underline{A[q] $\not=$ Tree} \\
					$\begin{array}{lll}
						H_s 			&=& \{s \vert s \in H \wedge D[s,p]\} \\
						(\neg D[q,p] \wedge (A[q] \not= Tree)) &\Rightarrow& A_c[s] = A[s] \Join A[q]\ \forall s \in H_s  	\\
					\end{array}$
				\end{tabular}
								\\
\hline							
\end{tabular}
\caption{Analysis Rules}
\label{fig:relwork_4}
\end{figure}

